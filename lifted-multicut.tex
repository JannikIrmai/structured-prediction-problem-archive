\section{Lifted Multicut}
\label{sec:lifted-multicut}

The lifted multicut problem~\cite{keuper2015efficient} is an extension of the multicut graph clustering problem to allow for connectivity priors.

\begin{definition}[Lifted Multicut]
Given two undirected weighted graphs $G=(V,E,c)$ and $G'=(V,E',c')$  called the base and lifted graph respectively the lifted multicut problem is to find a partition $\Pi = (\Pi_1,\ldots,\Pi_k)$ such that $\Pi_i \cap \Pi_j=\varnothing$ for $i \neq j$ and $\dot{\cup}_{i\in[k]} \Pi_j = V$ and \emph{each $\Pi_i$ is connected w.r.t. to the base graph $G$} where $k$ is computed as part of the optimization process.
The \emph{base cut} $y=\delta_G(\Pi_1,\ldots,\Pi_k)$, and \emph{lifted cut} $y' = \delta_{G'}(\Pi_1,\ldots,\Pi_k)$ induced by a decomposition is the subset of those base and lifted edges that straddle distinct clusters.

The space of all lifted multicuts is denoted as
\begin{equation}
  \mathcal{M'}_{G,G'} = \left\{ \begin{array}{c}\delta_G(V_1,\ldots,V_k)\\ \delta_{G'}(V_1,\ldots,V_k) \end{array} : 
\begin{array}{c}
  k \in \N \\
  V_1 \dot\cup \ldots \dot\cup V_k = V\\
  V_i \text{ connected in } G
\end{array} \right\}%\,.
\end{equation} 

The lifted minimum cost multicut problem is
\begin{equation}
  \label{eq:lifted-multicut}
  \min_{(y,y') \in \mathcal{M'}_{G,G'}} \la c, y \ra + \la c', y' \ra\,.
\end{equation} 
\end{definition}

The difference between multicut and lifted multicut is illustrated in Figure~\ref{fig:lifted-multicut}.

\begin{figure}[H]
    \begin{subfigure}{0.49\columnwidth}
    \begin{center}
        \includestandalone[width=\textwidth]{figures/lifted-multicut-1}
    \end{center}
        \caption{Multicut}
    \end{subfigure}
    \hfill
    \begin{subfigure}{0.49\columnwidth}
    \begin{center}
        \includestandalone[width=\textwidth]{figures/lifted-multicut-2}
    \end{center}
        \caption{Lifted Multicut. The base graph $G$ contains the grid edges.}
    \end{subfigure}
        \caption{
            Illustration of difference between multicut and lifted multicut.
            Dashed edges indicate cuts and solid intra-cluster ones.
            The lifted edge right forces the endpoints to be connected w.r.t.\ the base graph.
        }
        \label{fig:lifted-multicut}
\end{figure}
    
\subsection{File Format}
The lifted multicut problem is given in the text file format as follows:

\begin{fileformat}
LIFTED MULTICUT
BASE EDGES
(*$i_1$*) (*$j_1$*) (*$c_1$*)
.
.
.
(*$i_m$*) (*$j_m$*) (*$c_m$*)
LIFTED EDGES
(*$i'_1$*) (*$j'_1$*) (*$c'_1$*)
.
.
.
(*$i'_m$*) (*$j'_m$*) (*$c'_m$*) 
\end{fileformat}
where $E = \{i_1 j_1, \ldots, i_m j_m\}$ and $c_1,\ldots,c_m$ are the corresponding edge weights and
where $E' = \{i'_1 j'_1, \ldots, i'_m j'_m\}$ and $c'_1,\ldots,c'_m$ are the corresponding lifted edge weights.

\subsection{Datasets}
Four standard MOT benchmarks.
The MOT15/16/17 benchmarks \cite{MOTChallenge2015,MOT16} contain semi-crowded videos sequences filmed from a~static or a~moving camera. 
MOT20 \cite{MOTChallenge20} comprises crowded scenes with considerably higher number of  frames  and detections per frame

\subsubsection{Multi-Camera Multi Object Tracking}
\paragraph{Wildtrack}
\paragraph{Campus}
\paragraph{PETS-09}

\subsection{Algorithms}
\begin{description}
\item[GAEC\&KLj~\cite{keuper2015efficient}:] Greedy additive edge constraction (GAEC) and Kernighan \& Lin with joins (KLj) are extensions of the corresponding multicut algorithms to lifted multicut.
\item[Fusion Moves~\cite{beier2017multicut}:] Explore subspaces of lifted multicuts generated randomly via solving it with ILP-solvers.
\item[Balanced Edge Contraction~\cite{kardoost2018solving}:] BEC is an extension of the corresponding algorithm for multicut.
\item[Greedy Edge Fixation~\cite{levinkov2019comparative}:] GEF extension to lifted multicut.
\end{description}
