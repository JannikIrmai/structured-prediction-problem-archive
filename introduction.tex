\section{Structured Prediction}
Structured prediction is the task to predict structured solutions, which, in the following, will mean vectors that are subject to a number of constraints.
In particular, we restrict ourselves to integer linear programs, where all variables are binary subject to a number of constraints which commonly can be written as linear (in-)equalities, see Section~\ref{sec:ilp} below.
Due to the constraint structure finding even a feasible solution can be NP-hard. 
But even if the constraint structure admits easy construction of solutions, finding the best one w.r.t.\ an objective (often called energy in the literature) is typically NP-hard.
This and the large problem sizes typically occurring in machine learning and computer vision make structured prediction an interesting and difficult algorithmic challenge.

Over the years a large body of research has been devoted to efficient and scalable approximative and also optimal solvers for special subclasses of problems.
Somewhat less attention has been paid to generic, that is problem agnostic solvers.
In order to assess the empirical performance, datasets of instances in various problem classes have been  used.
It is the goal of the Structured Prediction Problem Archive to present and preserve interesting datasets coming from a large number of different problem classes together with references to algorithms which were proposed for their solution.
Unfortunately, in many previous works experiments were done on differing subsets of datasets, making comparison across different algorithms harder than it should be.
Similar projects have been pursued for Markov Random Fields~\cite{kappes2015comparative,szeliski2008comparative} with additional algorithm evaluations.

We hope that our project will make empirical testing of algorithms easier.
Our work provides a single point of reference from which problem instances can be obtained.
We hope that this will encourage algorithm testing on large corpora of instances.

\subsection{Integer Linear Programs}
\label{sec:ilp}
All structured prediction problems we collect can be written as integer linear programs (ILP).
An ILP is a linear minimization problem over integral variables that are subject to a number of linear constraints.
We constrain ourselves to $\{0,1\}$-valued variables.
Given an objective vector $c \in \R^n$, a constraint matrix $A \in \R^{n \times m}$ and a constraint right hand side $b \in \R^m$ we can write an ILP as

\begin{equation}
    \tag{ILP}
    \begin{array}{rl}
    \min_{x \in \{0,1\}^n}
    & \sum_{i\in[n]} c_i x_i \\
    \text{s.t.}
    & Ax \leq b \\
    \end{array}
\end{equation}
For greater flexibility we allow also equality constraints, i.e.\  $Ax = b$ or a mixture of inequalities and equalities.

\subsection{ILP File Format}
\label{sec:ilp-file-format}
Whenever applicable, e.g.\ when the number of constraints can be polynomially bounded, we store the problems in an LP file format that standard ILP solvers can read.
The format is structured as follows:

{\small
\begin{fileformat}
Minimize
(*$\sum\limits_i c_i var_i$*)
Subject to
ineq_id(*${}_1$*): (*$\sum\limits_{i}$*) (*$a_{1i} var_{i}$*)  (*$\{\leq|\geq|=\}$*) (*$b_1$*)
.
.
.
ineq_id(*${}_m$*): (*$\sum\limits_{i}$*) (*$a_{mi} var_{i}$*)  (*$\{\leq|\geq|=\}$*) (*$b_m$*)
Bounds
Binaries
(*$var_1$*)
.
.
.
(*$var_n$*)
End
\end{fileformat}
}

\subsection{Algorithms}
\begin{description}[style=unboxed]
\item[CPLEX~\cite{cplex}:] Classic leading commercial ILP solver from IBM.
\item[Gurobi~\cite{gurobi}:] Newer leading commercial ILP solver.
\item[Mosek~\cite{mosek}:] Another leading commercial ILP solver.
\item[SCIP~\cite{achterberg2009scip}:] Leading academic ILP and constraint integer solver.
\item[BDD Min-Marginal Averaging~\cite{lange2021efficient}:]
Decompose ILPs into subproblems represented by binary decision diagrams. Solve the resulting Lagrange decomposition by a sequential min-marginal averaging or with a parallel extension.
\item[Fast Discrete Optimization on GPU (FastDOG)~\cite{abbas2021fastdog}:] Extension of the BDD min-marginal averaging to a massively parallel GPU solver.
\end{description}
