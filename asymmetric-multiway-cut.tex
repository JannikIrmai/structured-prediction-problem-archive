\section{Asymmetric Multiway Cut}
The asymmetric multiway cut proposed in~\cite{kroeger2014asymmetric} is an extension of the multiway cut problem that allows for intra-class splits.

\begin{definition}[Asymmetric Multiway Cut]
Given a weighted graph $G=(V,E,c_E)$, $K$ classes, $P \subset [K]$ partitionable classes and node costs $c_V : [K]^V \rightarrow \R$ the asymmetric multiway cut problem is
\begin{equation}
    \begin{array}{rl}
    \min\limits_{\substack{x \in V \rightarrow [K]\\ y \in \mathcal{M}_G}}
    & \sum_{i \in V} c_V(i,x_i) + \sum_{ij \in E} c_E(e) \cdot y_e \\
    \text{s.t.}
    & x_i \neq x_j \Rightarrow y_e = 1\ \forall ij \in E \\
    & x_i = x_j \ \&\ x_i \notin P \Rightarrow y_e = 0\ \forall ij \in E \\
    \end{array}
\end{equation}
\end{definition}

An illustration of asymmetric multiway cut is given in Figure~\ref{fig:asymmetric-multiway-cut}.

\begin{figure}[H]
    \begin{center}
        \includegraphics[width=0.5\columnwidth]{figures/asymmetric-multiway-cut.tex}
    \end{center}
    \caption{Illustration of an asymmetric multiway cut with 4 classes and with class 3 having internal boundaries producing three clusters $3_A$, $3_B$ and $3_C$.}
    \label{fig:asymmetric-multiway-cut}
\end{figure}

\subsection{File Format}

\begin{fileformat}
ASYMMETRIC MULTIWAY CUT
PARTITIONABLE CLASSES
(*$p_1$*) ... (*$p_{\abs{P}}$*)

NODE COSTS
(*$c_1(1)$*) ... (*$c_1(K)$*) 
.
.
.
(*$c_{\abs{V}}(1)$*) ... (*$c_{\abs{V}}(K)$*)  

EDGE COSTS
(*$i_1$*) (*$j_1$*) (*$c_1$*)
.
.
.
(*$i_{\abs{E}}$*) (*$j_{\abs{E}}$*) (*$c_{\abs{E}}$*)
\end{fileformat}
where $P = \{p_1,\ldots,p_{\abs{P}}\}$ and
$E = \{i_1 j_1,\ldots,i_{\abs{E}} j_{\abs{E}}\}$.

\subsection{Datasets}

\subsubsection{Panoptic Segmentation}
In~\cite{abbas2021combinatorial} the asymmetric multiway cut solver was used for panoptic segmentation on the following two datasets.

\paragraph{Cityscapes}
Contains traffic related images from~\cite{cordts2016cityscapes} of resolution $1024 \times 2048$\footnote{\url{https://keeper.mpdl.mpg.de/f/63bfbea779d042a59a4d/?dl=1}} and $4 \times$ downsampled ones at resolution $256 \times 512$\footnote{\url{https://keeper.mpdl.mpg.de/f/956f3e74320a432295d9/?dl=1}}.
Images contain $21$ classes divided into $8$ \lq thing\rq\ and $11$ \lq stuff\rq\ classes.
$100$ easy and $100$ hard to segment images from the validation set were chosen for the benchmark. 

\paragraph{COCO}
Contains diverse images from~\cite{lin2014microsoft} of resolution $640 \times 480$\footnote{\url{https://keeper.mpdl.mpg.de/f/25852c04694047ce8e70/?dl=1}} and $4 \times$ downsampled to $160 \times 120$\footnote{\url{https://keeper.mpdl.mpg.de/f/dcdb2b8dc2f34b21a9bf/?dl=1}} with $133$ classes divided into $80$ \lq thing\rq\ and $53$ \lq stuff\rq\ classes.
$100$ easy and $100$ hard to segment images from the validation set were chosen for the benchmark. 

\subsection{Algorithms}
\begin{description}
\item[ILP~\cite{kroeger2014asymmetric}:] ILP description and cutting plane procedures for separating the defining inequalities.
\item[GAEC for AMWC~\cite{abbas2021combinatorial}:] Generalization of the multicut solver GAEC~\cite{keuper2015efficient} for asymmetric multiway cut.
\end{description}
